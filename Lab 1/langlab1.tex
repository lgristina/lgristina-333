%%%%%%%%%%%%%%%%%%%%%%%%%%%%%%%%%%%%%%%%%
%
% CMPT 435
% Lab Zero
%
%%%%%%%%%%%%%%%%%%%%%%%%%%%%%%%%%%%%%%%%%

%%%%%%%%%%%%%%%%%%%%%%%%%%%%%%%%%%%%%%%%%
% Short Sectioned Assignment
% LaTeX Template
% Version 1.0 (5/5/12)
%
% This template has been downloaded from: http://www.LaTeXTemplates.com
% Original author: % Frits Wenneker (http://www.howtotex.com)
% License: CC BY-NC-SA 3.0 (http://creativecommons.org/licenses/by-nc-sa/3.0/)
% Modified by Alan G. Labouseur  - alan@labouseur.com
%
%%%%%%%%%%%%%%%%%%%%%%%%%%%%%%%%%%%%%%%%%

%----------------------------------------------------------------------------------------
%	PACKAGES AND OTHER DOCUMENT CONFIGURATIONS
%----------------------------------------------------------------------------------------

\documentclass[letterpaper, 10pt]{article} 

\usepackage[english]{babel} % English language/hyphenation
\usepackage{graphicx}
\usepackage[lined,linesnumbered,commentsnumbered]{algorithm2e}
\usepackage{listings}
\usepackage{fancyhdr} % Custom headers and footers
\pagestyle{fancyplain} % Makes all pages in the document conform to the custom headers and footers
\usepackage{lastpage}
\usepackage{url}

\fancyhead{} % No page header - if you want one, create it in the same way as the footers below
\fancyfoot[L]{} % Empty left footer
\fancyfoot[C]{page \thepage\ of \pageref{LastPage}} % Page numbering for center footer
\fancyfoot[R]{}

\renewcommand{\headrulewidth}{0pt} % Remove header underlines
\renewcommand{\footrulewidth}{0pt} % Remove footer underlines
\setlength{\headheight}{13.6pt} % Customize the height of the header

%----------------------------------------------------------------------------------------
%	TITLE SECTION
%----------------------------------------------------------------------------------------

\newcommand{\horrule}[1]{\rule{\linewidth}{#1}} % Create horizontal rule command with 1 argument of height

\title{	
   \normalfont \normalsize 
   \textsc{CMPT 333 - Fall 2023 - Dr. Labouseur} \\[10pt] % Header stuff.
   \horrule{0.5pt} \\[0.25cm] 	% Top horizontal rule
   \huge Lab One -- Erlang's History \\     	    % Assignment title
   \horrule{0.5pt} \\[0.25cm] 	% Bottom horizontal rule
}

\author{Luca Gristina \\ \normalsize luca.gristina1@marist.edu}

\date{\normalsize\today} 	% Today's date.

\begin{document}

\maketitle % Print the title

%----------------------------------------------------------------------------------------
%   CONTENT SECTION
%----------------------------------------------------------------------------------------

% - -- -  - -- -  - -- -  -

\section{Lab 1 Questions}

\begin{enumerate}
    \item What is single assignment?
    \\\\
    It means that a variable can only be bound once.
    
    \item What's the difference between a bound and unbound variable?
    \\\\
    A bound variable has some sort of value attached to it while an unbound variable doesn't have a value.
    
    \item How does variable scope work in the Erlang environment?
    \\\\
    The scope of a variable is its function clause. Variables bound in a branch of an if, case, or receive expression must be bound in all branches to have a value outside the expression. Otherwise they are regarded as 'unsafe' outside the expression.

    \item Does Erlang implement mutable or immutable memory state? Why?
    \\\\
    Variables in Erlang are immutable. Once bound it can not be mutated nor become a different value. The immutable variables allow for increased efficiency within the language.

    \item Describe Erlang's memory management system?
    \\\\
    Each Erlang process has its own stack and heap which are allocated in the same memory block and grow toward each other. When the stack and the heap meet the garbage collector is triggered and memory is reclaimed.

    \item What does "Erlang" mean or stand for, if anything?
    \\\\
    It stands for Earl's Language.

    \item Contrast "soft real time" from "hard real time".
    \\\\
    "Soft" real time means that some requests can miss the deadline, while "Hard" real time means that all requests must be satisfied within a specified time.
    
    \item Why is Erlang so well suited for concurrency-oriented programming?
    \\\\
    It can handle large number of requests and processes at once. and it was made for building building systems that need to scale to handle large amounts of traffic.
    
    \item Explain Erlang's "let it crash" philosophy?
    \\\\
    It's an approach to error handling that seeks to preserve the integrity and reliability of a system by intentionally allowing certain faults to go unhandled.

    \item What's the difference between a tuple and a list?
    \\\\
    Tuples can be made up of multiple types, while a list has to be made up of one type.

    \item What's BEAM?
    \\\\
    BEAM is a register machine, where all instructions operate on named registers. Each register can contain any Erlang term such as an integer or a tuple.

    \item How can it be that we can create more Erlang "processes" than are allowed for in the operating system?
    \\\\
    Erlang is designed for massive concurrency. Erlang processes are lightweight (grow and shrink dynamically) with small memory footprint, fast to create and terminate, and the scheduling overhead is low.
    
\end{enumerate}

\vspace{2em}

\end{document}
