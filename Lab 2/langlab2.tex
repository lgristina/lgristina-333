%%%%%%%%%%%%%%%%%%%%%%%%%%%%%%%%%%%%%%%%%
%
% CMPT 333
% Lab Two
%
%%%%%%%%%%%%%%%%%%%%%%%%%%%%%%%%%%%%%%%%%

%%%%%%%%%%%%%%%%%%%%%%%%%%%%%%%%%%%%%%%%%
% Short Sectioned Assignment
% LaTeX Template
% Version 1.0 (5/5/12)
%
% This template has been downloaded from: http://www.LaTeXTemplates.com
% Original author: % Frits Wenneker (http://www.howtotex.com)
% License: CC BY-NC-SA 3.0 (http://creativecommons.org/licenses/by-nc-sa/3.0/)
% Modified by Alan G. Labouseur  - alan@labouseur.com
%
%%%%%%%%%%%%%%%%%%%%%%%%%%%%%%%%%%%%%%%%%

%----------------------------------------------------------------------------------------
%	PACKAGES AND OTHER DOCUMENT CONFIGURATIONS
%----------------------------------------------------------------------------------------

\documentclass[letterpaper, 10pt]{article} 

\usepackage[english]{babel} % English language/hyphenation
\usepackage{graphicx}
\usepackage[lined,linesnumbered,commentsnumbered]{algorithm2e}
\usepackage{listings}
\usepackage{fancyhdr} % Custom headers and footers
\pagestyle{fancyplain} % Makes all pages in the document conform to the custom headers and footers
\usepackage{lastpage}
\usepackage{url}

\fancyhead{} % No page header - if you want one, create it in the same way as the footers below
\fancyfoot[L]{} % Empty left footer
\fancyfoot[C]{page \thepage\ of \pageref{LastPage}} % Page numbering for center footer
\fancyfoot[R]{}
\setlength{\footskip}{75pt}

\renewcommand{\headrulewidth}{0pt} % Remove header underlines
\renewcommand{\footrulewidth}{0pt} % Remove footer underlines
\setlength{\headheight}{13.6pt} % Customize the height of the header

%----------------------------------------------------------------------------------------
%	TITLE SECTION
%----------------------------------------------------------------------------------------

\newcommand{\horrule}[1]{\rule{\linewidth}{#1}} % Create horizontal rule command with 1 argument of height

\title{	
   \normalfont \normalsize 
   \textsc{CMPT 333 - Fall 2023 - Dr. Labouseur} \\[10pt] % Header stuff.
   \horrule{0.5pt} \\[0.25cm] 	% Top horizontal rule
   \huge Lab Two -- Recursive Functions \\     	    % Assignment title
   \horrule{0.5pt} \\[0.25cm] 	% Bottom horizontal rule
}

\author{Luca Gristina \\ \normalsize luca.gristina1@marist.edu}

\date{\normalsize\today} 	% Today's date.

\begin{document}

\maketitle % Print the title

%----------------------------------------------------------------------------------------
%   CONTENT SECTION
%----------------------------------------------------------------------------------------

% - -- -  - -- -  - -- -  -
\setlength{\parindent}{20pt}
\section{Lab 2 Reflection}

Before starting this Lab, I was brainstorming ways that I could be able to solve this problem from what we learned in class and I was struggling to find a solution. I was originally thinking about it like I would in Java and was thinking to just iterate through each of the lists and populate them, until I was reading through chapter 4 in the book and came across list comprehensions. I found that they were way easier to use than mapping and made my function one line of code to create all of the lists. 

I started this lab by writing the Erlang code. I used recursion and list comprehensions to create and format the lists to achieve the desired output. When I first made the generateList function the output, although correct, was displayed wrong because of Erlang's pattern matching to the ascii values in the lists. So I had to recursively go back through the lists and format each of them to display the integer values. In the java version of this program that I wrote I made ArrayLists inside of Arraylists. I used nested for loops to calculate the needed values and populate the inner list with the integers and to populate the outer list with the inner lists. 

After completing this lab, I found Erlang to be quite powerful in its ability to complete this assignment. Despite this though, coding in Erlang first didn't seem to change my approach for the Java version. I still felt the best way to solve it was the way that I solved it using the nested for loops to populate the lists, but the coding in Erlang gave a good challenge for this lab.

\section{Transcript of Shell/Test Cases}

\begin{minipage}{0.6\textwidth}
\begin{verbatim}

1> c(lab2).
{ok,lab2}
2> lab2:test().
[14,28,42,56,70,84]
[13,27,41,55,69,83]
[12,26,40,54,68,82]
[11,25,39,53,67,81]
[10,24,38,52,66,80]
[9,23,37,51,65,79]
[8,22,36,50,64,78]
[7,21,35,49,63,77]
[6,20,34,48,62,76]
[5,19,33,47,61,75]
[4,18,32,46,60,74]
[3,17,31,45,59,73]
[2,16,30,44,58,72]
[1,15,29,43,57,71]
test passed
ok
3> lab2:test2().
[10,20,30,40,50]
[9,19,29,39,49]
[8,18,28,38,48]
[7,17,27,37,47]
[6,16,26,36,46]
[5,15,25,35,45]
[4,14,24,34,44]
[3,13,23,33,43]
[2,12,22,32,42]
[1,11,21,31,41]
test passed
ok
4> lab2:test3().
test passed
ok
5> lab2:test4().
test passed
ok
6> lab2:main(-5,10).
{error,"Invalid input: 
Please enter a non-negative integer."}
7> lab2:main("e",3).
{error,"Invalid input: 
Please enter a non-negative integer."}


\end{verbatim}
\end{minipage}
\hfill
\begin{minipage}{0.6\textwidth}
\begin{verbatim}
$ javac lab2.java
$ java lab2.java

----------------------------------
Test Case 1: example from lab page

[14, 28, 42, 56, 70, 84]
[13, 27, 41, 55, 69, 83]
[12, 26, 40, 54, 68, 82]
[11, 25, 39, 53, 67, 81]
[10, 24, 38, 52, 66, 80]
[9, 23, 37, 51, 65, 79]
[8, 22, 36, 50, 64, 78]
[7, 21, 35, 49, 63, 77]
[6, 20, 34, 48, 62, 76]
[5, 19, 33, 47, 61, 75]
[4, 18, 32, 46, 60, 74]
[3, 17, 31, 45, 59, 73]
[2, 16, 30, 44, 58, 72]
[1, 15, 29, 43, 57, 71]

----------------------------------
Test Case 2: More expected values

[10, 20, 30, 40, 50]
[9, 19, 29, 39, 49]
[8, 18, 28, 38, 48]
[7, 17, 27, 37, 47]
[6, 16, 26, 36, 46]
[5, 15, 25, 35, 45]
[4, 14, 24, 34, 44]
[3, 13, 23, 33, 43]
[2, 12, 22, 32, 42]
[1, 11, 21, 31, 41]

----------------------------------
Test Case 3: negative integer handling

Error: Please input a non-negative integer

----------------------------------
Test Case 4: Wanted to use a character to 
test my handling but it wouldn't compile
because its not an integer obviously

Error: Please input a non-negative integer

\end{verbatim}
\end{minipage}
\hfill
\vspace{2em}

\end{document}
